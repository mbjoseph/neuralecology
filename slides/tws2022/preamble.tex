
%Beamerの設定
\usetheme{Boadilla}

%Beamerフォント設定
\usepackage[T1]{fontenc}
% \usepackage{newtxtext}
% \usepackage{newtxmath} % TXフォント
\usepackage{sfmath}
% \usepackage[deluxe,uplatex]{otf} % 日本語多ウェイト化
% \usepackage[noto-otc]{pxchfon} %Noto CJK JPを指定
% \usepackage{pxjahyper} % PDF目次文字化け回避(platexでは不要)
\renewcommand{\familydefault}{\sfdefault}  % 英文をサンセリフ体に
\DeclareSymbolFont{operators}{OT1}{\sfdefault}{m}{n}
\SetSymbolFont{operators}{bold}{OT1}{\sfdefault}{b}{n}
\usefonttheme{structurebold} % タイトル部を太字
\setbeamerfont{alerted text}{series=\bfseries} % Alertを太字
\setbeamerfont{section in toc}{series=\mdseries} % 目次は太字にしない
\setbeamerfont{frametitle}{size=\Large} % フレームタイトル文字サイズ
\setbeamerfont{title}{size=\LARGE} % タイトル文字サイズ
\setbeamerfont{date}{size=\small}  % 日付文字サイズ
\usepackage{helvet}


% %Beamer色設定
\definecolor{UniBlue}{RGB}{37,140,34}
\definecolor{AlertOrange}{RGB}{255,76,0}
\definecolor{AlmostBlack}{RGB}{38,38,38}
\setbeamercolor{normal text}{fg=AlmostBlack}  % 本文カラー
\setbeamercolor{structure}{fg=UniBlue} % 見出しカラー
\setbeamercolor{block title}{fg=UniBlue!50!black} % ブロック部分タイトルカラー
\setbeamercolor{alerted text}{fg=AlertOrange} % \alert 文字カラー
\mode<beamer>{
    \definecolor{BackGroundGray}{RGB}{254,254,254}
    \setbeamercolor{background canvas}{bg=BackGroundGray} % スライドモードのみ背景をわずかにグレーにする
}

%フラットデザイン化
\setbeamertemplate{blocks}[rounded] % Blockの影を消す
\useinnertheme{circles} % 箇条書きをシンプルに
\setbeamertemplate{navigation symbols}{} % ナビゲーションシンボルを消す
\setbeamertemplate{footline}[frame number] % フッターはスライド番号のみ

%タイトルページ
\setbeamertemplate{title page}{%
    \vspace{2.5em}
    {\usebeamerfont{title} \usebeamercolor[fg]{title} \inserttitle \par}
    {\usebeamerfont{subtitle}\usebeamercolor[fg]{subtitle}\insertsubtitle \par}
    \vspace{1.5em}
    \begin{flushright}
        \usebeamerfont{author}\insertauthor\par
        \usebeamerfont{institute}\insertinstitute \par
        \vspace{3em}
        \usebeamerfont{date}\insertdate\par
        \usebeamercolor[fg]{titlegraphic}\inserttitlegraphic
    \end{flushright}
}

% Algorithm系
\usepackage{algorithm}
\usepackage[noend]{algorithmic}
\algsetup{linenosize=\color{fg!50}\footnotesize}
\renewcommand\algorithmicdo{:}
\renewcommand\algorithmicthen{:}
\renewcommand\algorithmicrequire{\textbf{Input:}}
\renewcommand\algorithmicensure{\textbf{Output:}}

% TikZ
\usepackage{tikz}
\usetikzlibrary{positioning,shapes,arrows}

% 定理
\theoremstyle{definition}
\newenvironment{mythm}{\begin{alertblock}{定理}}{\end{alertblock}} %自分の結果は赤色で表示

\AtBeginSection[]{
    \frame{\tableofcontents[currentsection, hideallsubsections]} %目次スライド
}


%タイトル
\title{Neural networks as drop-in function approximators}
\author{\textbf{Maxwell B. Joseph}}
\date{Nov 2022}
\institute{Natural Capital Exchange}

% custom diagrams
\usetikzlibrary{shapes.geometric, arrows}
\tikzstyle{whitebox} = [rectangle, rounded corners, minimum width=2cm, minimum height=1cm,text centered, draw=black, fill=white]
\tikzstyle{box} = [rectangle, rounded corners, minimum width=2cm, minimum height=1cm,text centered, draw=black, fill=UniBlue!50!white]
\tikzstyle{par} = [rectangle, rounded corners, minimum width=2cm, minimum height=1cm,text centered, draw=black, fill=AlertOrange!50!white]
\tikzstyle{arrow} = [thick,->,>=stealth]
\tikzstyle{neuron}=[circle,fill=black!25,minimum size=17pt,inner sep=0pt]

\def\layersep{2cm}


% custom footnote symbology
\usepackage[symbol]{footmisc}
\renewcommand{\thefootnote}{\fnsymbol{footnote}}
